\section{实验四:跨链路的多机通信实验(网络层)}

\subsection{功能要求}
本实验在复杂网状拓扑(Mesh)中,利用距离向量(Distance Vector)路由算法实现多机动态路由通信。核心目标包括:
\begin{itemize}
  \item 理解分布式路由网络的基本原理,每台 PC 作为对等的路由器节点。
  \item 实现邻居自动发现(Hello Protocol)与保持机制。
  \item 实现基于 Bellman-Ford 方程的距离向量路由算法,支持路由表动态收敛。
  \item 实现毒性逆转(Poison Reverse)优化,避免路由环路与"计数到无穷"问题。
  \item 支持链路上下线检测,路由失效后能触发超时检测并毒化路径,避免黑洞。
  \item 支持数据包跨两跳以上转发,用户可查询路由表并发送消息到任意 ID。
\end{itemize}
预期效果:网络自动形成可达路径,路由表在数个周期内收敛;链路断开后,相关目的开销置为不可达(999),并触发路由更新;恢复连接后重新收敛;数据包能够按照最优路径多跳转发。

\subsection{实现思路}
\subsubsection{硬件拓扑}
多串口互联形成包含至少一条 2 跳以上路径的网状结构。每台主机作为 \texttt{RouterNode} 同时连接 2 个及以上串口,构成多链路路由环境。实验采用 6 个节点(wolf、win、mac、A、B 及一个备用节点)的网状拓扑,某条路径可能为 \texttt{wolf → win → mac},另一条路径可能为 \texttt{wolf → A → B → mac},形成了多跳转发与多路径并存的复杂场景。每个节点需要配备多个 USB 转串口接口卡或多端口串口扩展卡,与邻居节点通过交叉连接建立点对点链路,所有链路波特率保持一致(默认 9600 bps)以简化配置。如图~\ref{fig:exp4_topology}所示,网状拓扑结构中各节点地位对等,不存在中心节点或单点故障风险,任意两节点间可能存在多条路径,距离向量算法需要在这种复杂环境下自动发现邻居、计算最优路径并动态适应链路变化,这为理解分布式路由协议的工作原理提供了直观的实验基础。

\begin{figure}[htbp]
\centering
\includegraphics[width=0.85\textwidth]{pic/exp6_topology.drawio.png}
\caption{实验四网状拓扑示意图}
\label{fig:exp4_topology}
\end{figure}

\subsubsection{功能流程}
整个距离向量路由系统的工作流程涵盖节点初始化、邻居发现、路由通告、数据转发、链路故障检测及路由表查询等多个环节,各环节相互配合形成了完整的分布式路由闭环。

\textbf{节点初始化} → 用户启动程序后,首先输入本机 ID(如 RouterA)并多选激活的串口(如 COM3、COM4),程序初始化路由表并将到自身的开销设为 0,为每个激活端口创建独立的监听线程以实现并发接收,同时启动三个定时任务线程。

\textbf{邻居发现}(周期:3 秒)→ 节点向所有激活端口广播 \texttt{HELLO|SenderID} 报文,邻居收到后立即记录发送端口与 ID 的映射关系,将直连邻居的开销置为 1,并更新邻居的最后活跃时间戳为当前时刻,用于后续超时检测。

\textbf{路由通告}(周期:5 秒)→ 节点向所有激活端口广播 \texttt{DV|SenderID|JSON\_Vector} 报文,其中 JSON\_Vector 包含本节点到全网各目的节点的开销信息,邻居收到后依据 Bellman-Ford 方程计算新开销(新开销 = 1 + 邻居声称的到目的开销),若新开销小于当前开销或发现新目的则立即更新路由表。

\textbf{毒性逆转优化} → 在构造每个出端口的 DV 通告时,若某目的的下一跳恰好是该出端口,则向该端口谎报该目的的开销为 999,从根本上避免了路由环路与无穷计数问题;同时检查邻居通告中缺失的目的地,若当前路由经由该邻居到达但邻居未通告,则将该目的开销置为 999 实现隐式毒化。

\textbf{数据转发} → 用户输入 \texttt{send <ID> <Msg>} 后,节点查路由表获取目的地的下一跳端口和下一跳 ID,将消息封装成 \texttt{DATA|Src|Dst|Payload} 格式并发送,中间节点收到数据包后检查目的地,若非自身则按路由表继续转发到下一跳,形成多跳路径;若目的地是自身则交付应用层,用户可在终端看到消息内容。

\textbf{邻居超时检测}(周期:1 秒)→ 检测邻居表中所有邻居的最后活跃时间,若某邻居距当前时刻超过 10 秒未发送 \texttt{HELLO},则从邻居表中移除该邻居,同时将所有经由该邻居的路由开销置为 999(毒化),触发立即的 DV 更新广播,确保故障链路不会继续被使用。

\textbf{路由表查询} → 用户输入 \texttt{table} 命令后,程序格式化打印当前路由表,显示目标 ID、开销、下一跳 ID 和接口端口,便于观察路由收敛状态、故障恢复过程以及毒化状态(开销 999)。

\textbf{优雅退出} → 用户输入 \texttt{exit} 命令后,程序首先标记运行状态为 false,所有后台任务线程检测到标志后逐个退出,随后主线程逐个关闭所有已打开的串口资源,最终程序优雅终止。

\subsubsection{协议定义}
定义三类网络层报文格式,字段以竖线分隔,行尾换行分帧:
\begin{itemize}
  \item \textbf{邻居发现报文}:\texttt{HELLO|SenderID}
  \begin{itemize}
    \item \texttt{HELLO}:报文类型标识
    \item \texttt{SenderID}:发送方节点 ID
  \end{itemize}
  \item \textbf{路由通告报文}:\texttt{DV|SenderID|JSON\_Vector}
  \begin{itemize}
    \item \texttt{DV}:报文类型标识
    \item \texttt{SenderID}:发送方节点 ID
    \item \texttt{JSON\_Vector}:JSON 格式的距离向量,例如 \texttt{\{"A": \{"cost": 1\}, "B": \{"cost": 2\}\}}
  \end{itemize}
  \item \textbf{数据传输报文}:\texttt{DATA|SrcID|DstID|Payload}
  \begin{itemize}
    \item \texttt{DATA}:报文类型标识
    \item \texttt{SrcID}:源节点 ID
    \item \texttt{DstID}:目的节点 ID
    \item \texttt{Payload}:消息内容
  \end{itemize}
\end{itemize}
路由表结构:\texttt{dest\_id → \{'cost': int, 'next\_hop\_port': port\_name, 'next\_hop\_id': id\}}

\subsection{模块设计}
\subsubsection{模块划分}
程序采用分层模块化设计,主要包含以下模块:
\begin{itemize}
  \item \textbf{路由节点核心}(\texttt{Code\_Refactored/Experiment4/router.py}):
  \begin{itemize}
    \item \texttt{RouterNode} 类:维护邻居表、路由表、端口管理、报文处理等核心逻辑。
    \item 实现 Bellman-Ford 更新算法、毒性逆转、邻居超时检测等机制。
  \end{itemize}
  \item \textbf{串口管理模块}:
  \begin{itemize}
    \item 为每个端口创建独立的监听线程(\texttt{\_listen\_port})。
    \item 使用线程锁(\texttt{port\_locks})保证并发写入安全。
  \end{itemize}
  \item \textbf{定时任务模块}:
  \begin{itemize}
    \item \texttt{\_task\_hello}:周期发送 \texttt{HELLO} 报文(每 3 秒)。
    \item \texttt{\_task\_broadcast\_dv}:周期广播距离向量(每 5 秒)。
    \item \texttt{\_task\_check\_timeout}:检测邻居超时(每 1 秒)。
  \end{itemize}
  \item \textbf{用户交互模块}:
  \begin{itemize}
    \item 支持 \texttt{table}、\texttt{send <ID> <Msg>}、\texttt{exit} 等命令。
  \end{itemize}
\end{itemize}

\begin{figure}[htbp]
\centering
% TODO: 插入模块调用关系图,展示RouterNode核心类管理多个串口监听线程,以及定时任务模块和用户交互模块的调用关系
\caption{实验四模块调用关系图}
\end{figure}

\subsubsection{接口定义}
关键接口定义如下:

\textbf{核心处理接口:}
\begin{itemize}
  \item \texttt{\_handle\_packet(raw, port\_source)}:解析报文并分派到 \texttt{HELLO/DV/DATA} 处理函数。
  \item \texttt{\_on\_recv\_hello(sender\_id, port)}:更新邻居表,将直连邻居开销置为 1。
  \item \texttt{\_on\_recv\_dv(sender\_id, dv\_json, port)}:解析邻居的距离向量,执行 Bellman-Ford 更新,若路由表变化则触发 DV 更新。
  \item \texttt{\_on\_recv\_data(src\_id, dst\_id, payload)}:若目的为本机则交付应用层,否则按路由表转发。
\end{itemize}

\textbf{路由更新接口:}
\begin{itemize}
  \item \texttt{\_send\_dv\_updates()}:为每个端口定制距离向量并广播,支持毒性逆转逻辑(若下一跳是该端口,则开销设为 999)。
\end{itemize}

\textbf{数据发送接口:}
\begin{itemize}
  \item \texttt{\_initiate\_send(target, msg)}:查表获取下一跳端口,封装数据包并发送。
  \item \texttt{\_send\_to\_port(port\_name, packet\_str)}:线程安全地发送数据到指定端口。
\end{itemize}

\textbf{用户命令接口:}
\begin{itemize}
  \item \texttt{\_print\_table()}:打印当前路由表,格式化显示目标、开销、下一跳 ID 和接口。
  \item \texttt{\_input\_loop()}:主循环,处理用户输入命令。
\end{itemize}

\subsection{功能实现}
\subsubsection{核心代码}
程序的核心实现要点如下:

\textbf{1. Bellman-Ford 路由更新算法}
收到邻居的距离向量后,依据 Bellman-Ford 方程更新路由表:
\begin{itemize}
  \item 新开销 = 1(到邻居的开销)+ 邻居到目标的开销
  \item 若新开销小于当前开销,或发现新目标,则更新路由表
  \item 若当前路由的下一跳是该邻居,且邻居通告中未包含该目标,则将该目标开销置为 999(隐式毒化)
\end{itemize}

\begin{verbatim}
def _on_recv_dv(self, sender_id, dv_json, port):
    try:
        neighbor_dv = json.loads(dv_json)
    except:
        return

    with self.rt_lock:
        updated = False
        
        # 1. 遍历邻居通告的所有目的地
        for dest, info in neighbor_dv.items():
            if dest == self.my_id: continue
            
            cost_neighbor_to_dest = info.get('cost', 999)
            new_cost = 1 + cost_neighbor_to_dest
            if new_cost > 999: new_cost = 999
            
            current_route = self.routing_table.get(dest)
            
            # 情况A: 发现新目标
            if not current_route:
                if new_cost < 999:
                    self.routing_table[dest] = {
                        'cost': new_cost,
                        'next_hop_port': port,
                        'next_hop_id': sender_id
                    }
                    updated = True
            
            # 情况B: 现有路由的下一跳就是该邻居
            elif current_route['next_hop_id'] == sender_id:
                if current_route['cost'] != new_cost:
                    current_route['cost'] = new_cost
                    updated = True
            
            # 情况C: 现有路由下一跳不是这个邻居,但提供了更短路径
            elif new_cost < current_route['cost']:
                self.routing_table[dest] = {
                    'cost': new_cost,
                    'next_hop_port': port,
                    'next_hop_id': sender_id
                }
                updated = True

        # 2. 检查是否有路由需要毒化
        for dest in list(self.routing_table.keys()):
            if dest == self.my_id: continue
            route = self.routing_table[dest]
            if route['next_hop_id'] == sender_id:
                if dest not in neighbor_dv:
                    if route['cost'] != 999:
                        route['cost'] = 999
                        updated = True

    if updated:
        self._send_dv_updates()
\end{verbatim}

\textbf{2. 毒性逆转(Poison Reverse)}
向邻居通告路由时,若某目的的下一跳是该邻居,则谎报开销为 999,避免路由环路。

\begin{verbatim}
def _send_dv_updates(self):
    with self.rt_lock:
        snapshot = {k:v.copy() for k,v in self.routing_table.items()}
    
    current_ports = list(self.active_ports.keys())
    
    for port_out in current_ports:
        neighbor_id = None
        with self.neighbors_lock:
            if port_out in self.neighbors:
                neighbor_id = self.neighbors[port_out]['id']
        
        # 构建针对该端口的DV
        custom_dv = {}
        for dest, info in snapshot.items():
            cost = info['cost']
            
            # 毒性逆转逻辑
            if info.get('next_hop_port') == port_out:
                cost = 999 
            
            custom_dv[dest] = {'cost': cost}
        
        dv_str = json.dumps(custom_dv)
        packet = f"{TYPE_DV}{SEPARATOR}{self.my_id}{SEPARATOR}{dv_str}"
        self._send_to_port(port_out, packet)
\end{verbatim}

\textbf{3. 邻居超时检测}
每秒检测邻居状态,若 10 秒内未收到 \texttt{HELLO},则移除邻居并毒化相关路由。

\begin{verbatim}
def _task_check_timeout(self):
    while self.running:
        now = time.time()
        timeout_ports = []
        
        with self.neighbors_lock:
            for port, info in self.neighbors.items():
                if now - info['last_seen'] > NEIGHBOR_TIMEOUT:
                    Logger.warning(f"[连接断开] 邻居 {info['id']} ({port}) 超时")
                    timeout_ports.append(port)
            for p in timeout_ports:
                del self.neighbors[p]
        
        if timeout_ports:
            with self.rt_lock:
                for dest, info in self.routing_table.items():
                    if info['next_hop_port'] in timeout_ports and dest != self.my_id:
                        info['cost'] = 999
        
        time.sleep(1)
\end{verbatim}

\textbf{4. 数据包转发}
收到数据包后,若目的为本机则交付,否则查表转发到下一跳端口。

\begin{verbatim}
def _on_recv_data(self, src_id, dst_id, payload):
    if dst_id == self.my_id:
        Logger.info(f">>> 收到消息 [{src_id}]: {payload}")
        print("> ", end="", flush=True)
        return
    
    with self.rt_lock:
        route = self.routing_table.get(dst_id)
        if route and route['cost'] < 999:
            next_port = route['next_hop_port']
            packet = f"{TYPE_DATA}{SEPARATOR}{src_id}{SEPARATOR}{dst_id}{SEPARATOR}{payload}"
            Logger.info(f"[转发] {src_id}->{dst_id} via {next_port}")
            self._send_to_port(next_port, packet)
        else:
            Logger.warning(f"[丢弃] 目标不可达: {dst_id} (From {src_id})")
\end{verbatim}

\subsubsection{实现效果}
在 6 节点网状拓扑中,实验各项功能均达到预期效果。

\textbf{路由收敛} → 所有节点启动后约 5-10 秒内路由表收敛完成,\texttt{table} 命令显示到全网各节点的最短开销及下一跳信息,例如 wolf 节点到 mac 节点的开销为 2,下一跳为 win,验证了 Bellman-Ford 算法的正确性和路由表学习机制的有效性。

\textbf{数据多跳转发} → 用户在 wolf 节点输入 \texttt{send mac Hello from wolf},消息经最优路径(wolf→win→mac)多跳转发到达目标节点,中间节点 win 输出转发日志(\texttt{[转发] wolf->mac via COM5}),mac 节点收到消息并在终端显示,证明了多跳转发机制的有效性。

\textbf{链路故障检测} → 拔掉 wolf 与 win 之间的串口线后,wolf 节点检测到 win 邻居超时(10 秒内未收到 HELLO),输出警告日志(\texttt{[连接断开] 邻居 win (COM3) 超时}),同时将所有经由 win 的路由开销置为 999,例如到 mac 的开销从 2 变为 999,此时发送消息给 mac 会被安全丢弃并告警(\texttt{[丢弃] 目标不可达: mac})。

\textbf{链路恢复自愈} → 重新连接 wolf 与 win 的串口线后,双方重新发送 HELLO 建立邻居关系,随后交换 DV 报文并相互更新路由表,路由表重新收敛,wolf 到 mac 的开销恢复为 2,消息能够正常转发,验证了路由协议的自愈能力和动态适应链路变化的能力。

\textbf{毒性逆转验证} → 通过日志观察 DV 通告内容,发现若某目的的下一跳是接收端口,则通告中该目的的开销显示为 999;例如 win 向 wolf 通告时,到 mac 的开销为实际开销 1,但向 mac 通告时,到 mac 的开销被毒化为 999,防止 mac 误以为可以通过 win 到达 mac 自身,有效避免了路由环路。

\begin{figure}[htbp]
\centering
\includegraphics[width=0.9\textwidth]{pic/exp4_results.png}
\caption{实验四实现效果截图}
\label{fig:exp4_results}
\end{figure}

\subsubsection{性能分析}
距离向量广播周期分别为 3 秒(\texttt{HELLO})与 5 秒(\texttt{DV}),路由表更新复杂度随目的数量线性增长,对小规模拓扑(6-10 节点)开销可忽略。串口 IO 采用行读取并加写锁,避免并发写冲突。邻居超时检测每秒执行一次,及时发现链路故障。毒性逆转有效避免了路由环路,但增加了通告开销。在 9600 bps 波特率下,路由收敛时间约 5-10 秒;提升至 115200 bps 可显著缩短收敛时间。

\paragraph{AI 提示词}
路由协议草拟阶段使用的提示词示例:\emph{"实现一个基于串口的距离向量路由器,支持 HELLO 邻居发现(每3秒)、DV 广播(每5秒)、Bellman-Ford 更新、毒性逆转、邻居超时检测(10秒),并提供 send/table 命令,使用多线程监听多个串口。"}