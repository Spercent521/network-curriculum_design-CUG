\section{实验二:双机通信实验(C/S模式)}

\subsection{功能要求}
本实验在实验一的基础上,扩展为两台设备的交叉串口通信链路,实现典型的客户端/服务器(C/S)交互模式。核心目标包括:
\begin{itemize}
  \item 掌握双机串口通信的交叉连接方法,理解 TX/RX 交叉接线与共地的重要性。
  \item 构建服务器端程序,实现常驻监听、命令解析与响应返回功能。
  \item 构建客户端程序,实现主动连接、命令发送与响应接收功能。
  \item 实现基础命令集:\texttt{HELLO} 握手、\texttt{TIME} 查询服务器时钟、\texttt{ECHO <msg>} 消息回显、\texttt{CALC <expr>} 远程计算表达式、\texttt{QUIT} 结束会话。
  \item 收发双方在终端显示交互内容,并能通过调试模式查看原始报文的十六进制表示,便于问题排查。
\end{itemize}
预期效果:客户端发送的命令能够被服务器准确解析并返回正确结果;关闭命令能够使双方有序退出;整个过程无乱码、无阻塞。

\subsection{实现思路}
\subsubsection{硬件拓扑}
两台主机通过 USB 转串口模块实现交叉连接,形成点对点双向通信链路。具体连接方式为:计算机 A 的 TX(发送)引脚连接到计算机 B 的 RX(接收)引脚,计算机 B 的 TX(发送)引脚连接到计算机 A 的 RX(接收)引脚,同时两台计算机的 GND(地线)引脚相连以确保共地,避免电平不匹配导致的通信异常。双方串口参数必须保持严格一致,默认配置为 9600 bps、8N1 格式,但程序支持通过交互式菜单调整波特率、数据位、停止位、校验位等参数,以适应不同的硬件环境与传输需求。如图~\ref{fig:exp2_topology}所示,交叉连接的拓扑结构简单直观,既避免了单机回环的局限性,又为后续多机组网实验奠定了基础,是理解双向串口通信的关键环节。

\begin{figure}[htbp]
\centering
\includegraphics[width=0.7\textwidth]{pic/exp2_topology.drawio.png}
\caption{实验二硬件拓扑示意图}
\label{fig:exp2_topology}
\end{figure}

\subsubsection{功能流程}
客户端与服务器的交互流程遵循典型的请求-响应模式。整个过程可以描述为以下阶段的有序展开:

\textbf{服务器启动与监听} → 服务器程序启动后,交互式选择串口、设置波特率与格式参数 → 打开串口连接 → 启动后台监听线程进入被动监听状态,持续读取串口数据以捕获客户端请求。

\textbf{客户端启动与连接} → 客户端选择串口并设置与服务器一致的波特率与格式 → 打开串口 → 自动启动后台接收线程 → 立即发送初始 \texttt{HELLO} 命令建立连接并确认服务器就绪。

\textbf{命令交互阶段} → 客户端在命令行输入命令(如 \texttt{TIME}、\texttt{ECHO Hello}) → 程序通过串口发送该命令 → 服务器接收线程按行读取数据后调用命令处理函数 → 根据命令类型解析参数并生成对应响应字符串 → 服务器将响应发送回客户端。

\textbf{响应接收与显示} → 客户端后台接收线程读取服务器响应 → 立即在终端显示给用户,实现实时交互效果。支持的命令包括 \texttt{TIME}(查询时间)、\texttt{ECHO}(消息回显)、\texttt{CALC}(远程计算)等,每条命令都获得对应的响应。

\textbf{退出流程} → 客户端发送 \texttt{QUIT} 命令 → 服务器返回告别信息并结束监听循环 → 双方同时关闭串口 → 所有线程终止,程序优雅退出。

整个交互过程中,客户端和服务器分工明确:客户端负责用户输入与响应展示,服务器负责命令解析与数据处理,通过串口作为物理传输媒介实现端到端的应用层通信。

\subsubsection{协议定义}
采用基于文本行的简单应用层协议:
\begin{itemize}
  \item \textbf{编码格式}:UTF-8 编码,支持中文和英文字符。
  \item \textbf{分帧方式}:以换行符(\texttt{\textbackslash n})作为请求和响应的分帧边界,有效避免粘包和半包问题。
  \item \textbf{请求格式}:\texttt{<COMMAND> [参数]},命令关键字不区分大小写。支持的命令包括:
  \begin{itemize}
    \item \texttt{HELLO}:建立连接问候,无参数
    \item \texttt{TIME}:查询服务器当前时间,无参数
    \item \texttt{ECHO <msg>}:回显消息,参数为任意文本
    \item \texttt{CALC <expr>}:计算表达式,参数为 Python 合法表达式(如 \texttt{2+3*4})
    \item \texttt{QUIT}:断开连接并退出,无参数
  \end{itemize}
  \item \textbf{响应格式}:统一以 \texttt{SERVER: <响应内容>} 开头,便于客户端识别和解析。
  \item \textbf{调试模式}:双方均支持十六进制调试输出,显示原始报文的十六进制表示,便于查错。
\end{itemize}

\subsection{模块设计}
\subsubsection{模块划分}
程序采用 C/S 架构,主要包含三个模块:
\begin{itemize}
  \item \textbf{串口工具层}(\texttt{Code\_Refactored/utils.py}):复用实验一的串口工具,提供端口扫描、串口创建、格式选择(\texttt{choose\_serial\_format})等通用功能。
  \item \textbf{服务器端}(\texttt{Code\_Refactored/Experiment2/server.py}):定义 \texttt{SerialServer} 类,持有串口对象、接收线程、运行状态标志。核心方法包括命令解析(\texttt{process\_request})和响应发送。
  \item \textbf{客户端}(\texttt{Code\_Refactored/Experiment2/client.py}):定义 \texttt{SerialClient} 类,负责命令输入、请求封装发送、接收显示。内置帮助系统,支持用户查询可用命令。
\end{itemize}

\begin{figure}[htbp]
\centering
% TODO: 插入模块调用关系图,展示客户端和服务器分别调用utils层,以及各自内部的方法调用关系
\caption{实验二模块调用关系图}
\end{figure}

\subsubsection{接口定义}
关键接口定义如下:

\textbf{服务器端接口(SerialServer 类):}
\begin{itemize}
  \item \texttt{open\_port(port, baudrate, bytesize, stopbits, parity)}:依据用户选择的串口格式打开串口。
  \item \texttt{close\_port()}:停止接收线程,关闭串口,释放资源。
  \item \texttt{start\_server()}:启动监听线程,进入服务状态。
  \item \texttt{process\_request(request)}:解析请求字符串,根据命令类型返回响应字符串及退出标志。支持 \texttt{HELLO}、\texttt{TIME}、\texttt{ECHO}、\texttt{CALC}、\texttt{QUIT} 命令。
  \item \texttt{receive\_worker()}:后台接收线程的工作函数,按行读取串口数据,调用 \texttt{process\_request} 处理并发送响应。
\end{itemize}

\textbf{客户端接口(SerialClient 类):}
\begin{itemize}
  \item \texttt{open\_port(port, baudrate, bytesize, stopbits, parity)}:打开串口,自动启动接收线程。
  \item \texttt{close\_port()}:停止接收线程,关闭串口。
  \item \texttt{send\_request(request)}:发送请求到服务器,自动补全行尾换行符,确保服务器能够按行读取。
  \item \texttt{\_receive\_worker()}:后台接收线程的工作函数,读取服务器响应并显示在终端。
\end{itemize}

\textbf{工具层接口(utils.py):}
\begin{itemize}
  \item \texttt{select\_serial\_port(prompt)}:扫描可用串口,以数字菜单形式供用户选择。
  \item \texttt{create\_serial\_connection(port, baudrate, ...)}:创建并打开串口连接。
  \item \texttt{choose\_serial\_format()}:交互式选择串口格式(数据位、停止位、校验位),返回格式参数和标签。
\end{itemize}

\subsection{功能实现}
\subsubsection{核心代码}
程序的核心实现要点如下:

\textbf{1. 服务器端命令解析}
\texttt{process\_request} 方法使用字符串匹配解析命令,根据不同命令类型返回相应响应。\texttt{CALC} 命令使用 Python 内置 \texttt{eval} 函数计算表达式,异常时返回错误提示。

\begin{verbatim}
def process_request(self, request):
    request_str = request.decode('utf-8', errors='ignore').strip()
    
    if request_str.upper().startswith("HELLO"):
        response = "SERVER: Hello, Client! Connection established."
    elif request_str.upper().startswith("TIME"):
        response = f"SERVER: Current time is {time.strftime('%Y-%m-%d %H:%M:%S')}"
    elif request_str.upper().startswith("ECHO"):
        echo_content = request_str[5:].strip()
        response = f"SERVER: ECHO - {echo_content}"
    elif request_str.upper().startswith("CALC"):
        try:
            expr = request_str[5:].strip()
            result = eval(expr)
            response = f"SERVER: CALC - {expr} = {result}"
        except:
            response = "SERVER: ERROR - Invalid calculation expression"
    elif request_str.upper().startswith("QUIT"):
        response = "SERVER: Goodbye!"
        return response, True  # 返回退出标志
    else:
        response = f"SERVER: Unknown command '{request_str}'"
    
    return response, False
\end{verbatim}

\textbf{2. 客户端自动连接}
客户端在启动后自动发送 \texttt{HELLO} 命令建立连接,验证服务器就绪。

\begin{verbatim}
Logger.info("正在连接服务器...")
time.sleep(0.5)  # 等待服务器就绪
client.send_request("HELLO")
\end{verbatim}

\textbf{3. 调试输出}
双方均提供十六进制调试输出功能,便于查错。调试输出包含时间戳、方向、长度、原始内容和十六进制表示。

\begin{verbatim}
def _log(self, direction, payload):
    ts = time.strftime('%H:%M:%S')
    if not isinstance(payload, bytes):
        payload = str(payload).encode('utf-8', errors='ignore')
    raw = repr(payload)
    hex_str = ' '.join(f"{b:02X}" for b in payload)
    print(f"[DEBUG {ts}] {direction}: len={len(payload)} raw={raw} hex={hex_str}")
\end{verbatim}

\subsubsection{实现效果}
在 9600 bps 波特率下,C/S 交互各项功能均运行稳定且响应及时。具体实现结果如下:

\textbf{连接建立验证} → 客户端启动后自动发送 \texttt{HELLO},服务器立即返回"SERVER: Hello, Client! Connection established.",连接建立成功并输出日志提示。

\textbf{时间查询功能} → 客户端发送 \texttt{TIME} 命令,服务器实时调用系统时间函数并返回格式化的当前时间(如"SERVER: Current time is 2026-01-25 09:30:45"),验证了服务器端的动态数据获取能力。

\textbf{消息回显测试} → 客户端发送 \texttt{ECHO Hello World},服务器解析参数并返回"SERVER: ECHO - Hello World",验证了参数解析与内容透传的正确性。

\textbf{远程计算功能} → 客户端发送 \texttt{CALC 2+3*4},服务器调用 \texttt{eval} 函数计算表达式并返回"SERVER: CALC - 2+3*4 = 14",展示了应用层协议的可扩展性与动态计算能力。

\textbf{优雅退出流程} → 客户端发送 \texttt{QUIT},服务器返回"SERVER: Goodbye!"并退出监听循环,双方同时关闭串口,整个过程无阻塞、无异常。

\textbf{性能与波特率} → 在 9600 bps 下,每条命令的端到端延迟约 50-100ms(取决于串口缓冲区与系统调度),若升级至 115200 bps,响应速度显著加快,延迟可缩短至 5-10ms 以内,端到端延迟可忽略不计。

\begin{figure}[htbp]
\centering
\includegraphics[width=0.9\textwidth]{pic/exp2_results.png}
\caption{实验二实现效果截图}
\label{fig:exp2_results}
\end{figure}

\subsubsection{性能分析}
通信量小、协议简单,单线程轮询读取即可满足需求。行分隔符有效解决了粘包问题,唯一瓶颈为串口物理速率,对本实验不构成限制。在 9600 bps 下,端到端延迟可忽略不计;若提升至 115200 bps,响应速度显著加快。调试输出在查错时启用,正常使用时可关闭以减少日志开销。

\paragraph{AI 提示词}
生成 C/S 串口交互样例时的提示词示例:\emph{"用 Python 编写一个串口客户端和服务器程序,采用文本行协议,支持 HELLO/TIME/ECHO/CALC/QUIT 命令,带调试日志(十六进制输出)和交互式串口参数配置,实现优雅退出机制。"}