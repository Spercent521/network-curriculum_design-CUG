\section{实验三:简单拓扑的多机通信实验(链路层)}

\subsection{功能要求}
本实验在实验二的基础上,扩展为多机通信,构建一个简单的树形拓扑网络,实现数据链路层的帧转发功能。核心目标包括:
\begin{itemize}
  \item 理解树形拓扑的网络结构,掌握星型物理连接与树形逻辑拓扑的概念。
  \item 实现根节点(Root/Switch)的转发功能,维护设备 ID 到串口端口的转发表。
  \item 实现叶子节点(Leaf)的通信功能,封装/解析数据帧,仅接收匹配自身 ID 的消息。
  \item 支持根节点主动下行发送消息,验证双向链路的正确性。
  \item 实现基于 ID 的消息寻址,任意两叶子节点间通过根节点中转完成通信。
\end{itemize}
预期效果:不同叶子节点之间的消息能够经根节点正确中转;根节点能够主动发送消息给任意叶子节点;未知目标 ID 的消息被根节点丢弃并给出警告提示。

\subsection{实现思路}
\subsubsection{硬件拓扑}
采用一台根节点主机连接多根串口线,分别与多个叶子节点的单串口相连,形成星型物理连接、树形逻辑拓扑。所有链路波特率保持一致(默认 9600 bps)。示例拓扑中,根节点主机(PC1)配备多个 USB 转串口接口卡或多端口串口扩展卡,分别连接三个叶子节点(ID2、ID3、ID4),每个叶子节点仅配备一个串口与根节点相连。若 ID2 想发消息给 ID3,路径为:\texttt{ID2 → Root → ID3},所有跨叶子的通信必须经根节点中转,因此根节点需要维护设备 ID 到串口端口的转发表以实现正确的帧路由。树形拓扑结构层次清晰,根节点居于中心位置负责转发,叶子节点作为终端设备仅需实现点对点通信逻辑,这种设计简化了叶子节点的复杂度,同时为理解交换机或集线器的工作原理提供了直观的实验基础。

\begin{figure}[htbp]
\centering
\includegraphics[width=0.8\textwidth]{pic/exp3_topology.drawio.png}
\caption{实验三硬件拓扑示意图}
\label{fig:exp3_topology}
\end{figure}

\subsubsection{功能流程}
整个树形拓扑的多机通信流程涵盖根节点与叶子节点的初始化、帧封装与解析、查表转发及资源清理等多个环节,形成了完整的链路层通信闭环。

\textbf{根节点启动与配置} → 根节点程序启动后,通过交互式界面逐个添加连接的串口端口 → 为每个端口绑定对应的远端设备 ID(如 COM4 连接 ID2、COM5 连接 ID3) → 程序生成转发表(ID → Port 映射) → 为每个端口创建独立的监听线程,确保多端口并发接收而互不阻塞。

\textbf{叶子节点启动} → 叶子节点选择连接到根节点的串口 → 设置本机 ID(如 ID2) → 启动接收线程进入就绪状态,等待接收来自根节点或其他叶子的消息。

\textbf{叶子节点发送} → 用户输入目标 ID 和消息内容 → 程序封装为数据帧(格式:\texttt{SRC\_ID|DST\_ID|PAYLOAD}) → 通过串口发送给根节点。

\textbf{根节点接收与转发} → 根节点的端口监听线程收到帧 → 解析源 ID 和目的 ID → 执行转发决策:若目的 ID 为根节点自身(ROOT)则打印消息;若目的 ID 在转发表中存在则查表获取对应端口并转发该帧,同时输出转发日志;若目的 ID 不存在则丢弃帧并输出警告日志。

\textbf{目标叶子接收} → 目标叶子节点接收帧 → 解析目的 ID → 若匹配自身 ID 则显示消息内容,否则忽略该帧。

\textbf{根节点主动发送} → 根节点可主动向任意 ID 发送消息 → 查找转发表 → 直接发送到对应端口,验证双向链路的正确性。

\textbf{退出清理} → 用户输入 \texttt{exit} 命令 → 所有端口监听线程停止运行 → 串口逐个关闭 → 程序优雅退出。

\subsubsection{协议定义}
链路层采用自定义帧格式,实现简单的 ID 寻址:
\begin{itemize}
  \item \textbf{帧格式}:\texttt{SRC\_ID|DST\_ID|PAYLOAD},字段以竖线(\texttt{|})分隔,以换行符(\texttt{\textbackslash n})作为帧边界。
  \item \textbf{字段说明}:
  \begin{itemize}
    \item \texttt{SRC\_ID}:发送方设备 ID(如 ID2)
    \item \texttt{DST\_ID}:接收方设备 ID(如 ID3、ROOT)
    \item \texttt{PAYLOAD}:消息内容(UTF-8 编码)
  \end{itemize}
  \item \textbf{寻址机制}:根节点根据 \texttt{DST\_ID} 查转发表,若匹配则转发,否则丢弃。可选支持 \texttt{DST=BROADCAST} 广播模式。
  \item \textbf{编码格式}:UTF-8 编码,支持中文和英文字符。
\end{itemize}

\subsection{模块设计}
\subsubsection{模块划分}
程序采用模块化设计,主要包含三个模块:
\begin{itemize}
  \item \textbf{串口工具层}(\texttt{Code\_Refactored/utils.py}):复用实验一的串口工具,提供端口扫描、串口创建、格式选择等通用功能。
  \item \textbf{根节点}(\texttt{Code\_Refactored/Experiment3/root.py}):
  \begin{itemize}
    \item \texttt{RootNode} 类:维护转发表(\texttt{routing\_table}),提供端口添加、消息处理、主动发送等功能。
    \item \texttt{PortListener} 线程:为每个端口创建独立的监听线程,负责读取串口数据并回调到 \texttt{RootNode.handle\_message}。
  \end{itemize}
  \item \textbf{叶子节点}(\texttt{Code\_Refactored/Experiment3/leaf.py}):
  \begin{itemize}
    \item \texttt{LeafNode} 类:负责帧封装/解析、发送和接收消息,仅显示目的为自身 ID 的消息。
  \end{itemize}
\end{itemize}

\begin{figure}[htbp]
\centering
% TODO: 插入模块调用关系图,展示RootNode管理多个PortListener线程,LeafNode直接调用utils层
\caption{实验三模块调用关系图}
\end{figure}

\subsubsection{接口定义}
关键接口定义如下:

\textbf{根节点接口(RootNode 类):}
\begin{itemize}
  \item \texttt{add\_port(port, baudrate, connected\_id)}:为指定端口创建 \texttt{PortListener} 监听线程,并将 ID 到端口的映射写入转发表。
  \item \texttt{handle\_message(raw\_data, source\_port)}:处理接收到的消息,解析帧(\texttt{SRC|DST|PAYLOAD}),根据目的 ID 转发或丢弃。
  \item \texttt{send\_message(target\_id, msg)}:根节点主动发送消息,封装为帧并查表发送到目标端口。
  \item \texttt{stop()}:停止所有端口监听线程,关闭所有串口。
\end{itemize}

\textbf{端口监听接口(PortListener 线程):}
\begin{itemize}
  \item \texttt{run()}:监听线程的主循环,持续读取串口数据,回调到 \texttt{RootNode.handle\_message}。
  \item \texttt{send(data)}:向端口发送数据(线程安全)。
  \item \texttt{stop()}:停止监听线程,关闭串口。
\end{itemize}

\textbf{叶子节点接口(LeafNode 类):}
\begin{itemize}
  \item \texttt{connect(port, baudrate, my\_id)}:连接到根节点,设置本机 ID,启动接收线程。
  \item \texttt{send\_message(target\_id, message)}:封装帧(\texttt{SRC|DST|MSG})并发送。
  \item \texttt{\_process\_frame(raw\_data)}:处理接收到的帧,若目的 ID 匹配自身则显示消息。
  \item \texttt{stop()}:停止接收线程,关闭串口。
\end{itemize}

\subsection{功能实现}
\subsubsection{核心代码}
程序的核心实现要点如下:

\textbf{1. 根节点转发表与消息处理}
根节点维护两个核心数据结构:
\begin{itemize}
  \item \texttt{listeners}:端口名称 → \texttt{PortListener} 对象的映射
  \item \texttt{routing\_table}:设备 ID → 端口名称的映射
\end{itemize}
\texttt{handle\_message} 方法解析帧后,根据目的 ID 查表转发。

\begin{verbatim}
def handle_message(self, raw_data, source_port):
    parts = raw_data.split(SEPARATOR, 2)
    if len(parts) != 3:
        Logger.debug(f"[收到畸形帧] {raw_data} 来自 {source_port}")
        return

    src_id, dst_id, payload = parts
    print(f"[RECV] {src_id} -> {dst_id} : {payload} (来自 {source_port})")

    # 判断是否发给自己
    if dst_id == self.my_id:
        print(f"  >>> 收到发给自己的消息: {payload}")
        return

    # 查表转发
    if dst_id in self.routing_table:
        target_port = self.routing_table[dst_id]
        if target_port == source_port:
            Logger.warning(f"  [警告] 目标端口与源端口相同,丢弃")
            return
        if target_port in self.listeners:
            print(f"  >>> 转发至端口 {target_port} (目标: {dst_id})")
            self.listeners[target_port].send(raw_data)
        else:
            Logger.error(f"  [ERROR] 目标端口 {target_port} 未在监听列表")
    else:
        Logger.warning(f"  [丢弃] 未知目标ID: {dst_id} (转发表中不存在)")
\end{verbatim}

\textbf{2. 端口监听线程}
每个端口对应一个独立的 \texttt{PortListener} 线程,避免阻塞主线程。

\begin{verbatim}
class PortListener(threading.Thread):
    def run(self):
        self.ser = create_serial_connection(self.port, self.baudrate, timeout=0.1)
        if self.ser:
            self.running = True
            Logger.info(f"[{self.port}] 端口已打开,连接设备: {self.user_id}")
            
            while self.running:
                if self.ser.in_waiting:
                    try:
                        line = self.ser.readline().decode('utf-8', errors='ignore').strip()
                        if line:
                            self.callback(line, self.port)
                    except Exception as e:
                        Logger.error(f"[{self.port}] 读取错误: {e}")
                time.sleep(0.01)
\end{verbatim}

\textbf{3. 叶子节点帧解析与过滤}
叶子节点仅显示目的 ID 匹配自身的消息。

\begin{verbatim}
def _process_frame(self, raw_data):
    parts = raw_data.split(SEPARATOR, 2)
    if len(parts) != 3:
        return

    src_id, dst_id, payload = parts

    if dst_id == self.my_id:
        print(f"\n[收到消息] 来自 {src_id}: {payload}")
        print(f"> ", end="", flush=True)
    elif dst_id == "BROADCAST": # 可选:支持广播
         print(f"\n[收到广播] 来自 {src_id}: {payload}")
         print(f"> ", end="", flush=True)
    else:
        # 目标不是自己,忽略
        pass
\end{verbatim}

\subsubsection{实现效果}
在由 1 个根节点和 3 个叶子节点(ID2、ID3、ID4)组成的树形拓扑中,实验各项功能均达到预期效果。

\textbf{叶子间通信验证} → ID2 发送消息给 ID3,根节点实时打印转发日志:"\texttt{[RECV] ID2 -> ID3 : Hello from ID2 (来自 COM4) >>> 转发至端口 COM5 (目标: ID3)}",ID3 正确收到并在终端显示消息,验证了基于 ID 的查表转发机制。

\textbf{根节点主动发送} → 根节点输入"ID4 Hello from ROOT",消息直接转发到连接 ID4 的端口,ID4 收到消息并显示,证明双向链路工作正常。

\textbf{未知目标丢弃} → ID2 发送消息给不存在的 ID5,根节点输出警告日志:"\texttt{[丢弃] 未知目标ID: ID5 (转发表中不存在)}",消息被正确丢弃而不会造成网络混乱。

\textbf{多叶子并发通信} → ID2、ID3、ID4 同时向不同目标发送消息,根节点的多线程监听机制确保了并发转发的正确性,各叶子节点均能收到发往自己的消息而忽略其他消息。

\textbf{转发日志与调试} → 根节点的转发日志清晰记录了每一帧的来源端口、目的 ID 和转发决策,包括转发成功、未知目标丢弃等详细信息,便于调试与问题定位。

\textbf{链路鲁棒性} → 即使在并发转发高峰期,根节点依然能准确转发消息,无错误、无遗漏,各叶子间通过根节点中转完成可靠的双向通信。

\begin{figure}[htbp]
\centering
\includegraphics[width=0.9\textwidth]{pic/exp3_results.png}
\caption{实验三实现效果截图}
\label{fig:exp3_results}
\end{figure}

\subsubsection{性能分析}
链路层仅做一次查表与串口写入,耗时主要取决于串口速率。根节点采用多线程监听,每个端口一个独立线程,CPU 占用低,各端口通信互不阻塞。瓶颈在于物理链路数量与波特率,在 9600 bps 下,并发通信时可能出现缓冲区拥塞,可通过提高波特率或优化调度策略改善。

\paragraph{AI 提示词}
设计树形转发逻辑时的提示词示例:\emph{"编写 Python 串口根节点程序,维护设备 ID 到串口的转发表,收到 SRC|DST|PAYLOAD 格式数据后按目的 ID 查表转发,使用多线程监听多个串口,支持主动发送消息和未知目标丢弃功能。"}