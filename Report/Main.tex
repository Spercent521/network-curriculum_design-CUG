% !TEX program = xelatex
% !TEX bib_program = biber

\documentclass{ctexart}

% 字体与中文设置
\setmainfont{Times New Roman}
\setCJKmainfont{Microsoft YaHei}

% 页面与格式设定
\usepackage[top=2.5cm, bottom=2.5cm, left=3cm, right=3cm]{geometry}
\usepackage{setspace}
\usepackage{titlesec}
\usepackage{fancyhdr}
\usepackage{graphicx}
\usepackage{caption}

% 修复 fancyhdr 警告:增加页眉高度
\setlength{\headheight}{13pt}

% 行距
\setstretch{1.5}

% 标题设定
\titleformat{\section}{\large\bfseries}{\thesection}{1em}{}
\titleformat{\subsection}{\normalsize\bfseries}{\thesubsection}{1em}{}
\titleformat*{\subsubsection}{\normalsize}

% 页眉页脚
\pagestyle{fancy}
\fancyhf{}
\fancyhead[L]{计算机网络构建系列实验报告}
\fancyhead[R]{\thepage}

% 参考文献
\usepackage[backend=biber, style=numeric, sorting=none]{biblatex}
\addbibresource{refs.bib}

\begin{document}

\begin{center}
  % --- 标题 ---
  {\LARGE \textbf{计算机网络构建系列实验项目报告}}\\[1.5em]

  {\large 姓名: 孙泽龙 \hspace{1em} 学号: 20241003223}\\[1.0em]
  {\small
    班级: 191242 \\
  }
\end{center}

\vspace{1em}
\begin{abstract}
本报告详细记录了“计算机网络构建系列实验项目”的实施过程与结果。项目共包含6个递进式实验,从底层的单机串口通信、双机C/S模式通信,到链路层的简单拓扑多机通信、网络层的跨链路通信,最终实现运输层的可靠传输及应用层的网络管理功能(ping与traceroute)。通过这一系列实验,深入理解并实践了计算机网络的分层架构、协议设计与核心算法实现。
\end{abstract}
\vspace{1em}

% 正文部分:按实验章节组织
\section{实验一:单机串口通信}

\subsection{功能要求}
% 简述对应实验的核心目标、需实现的功能点、预期效果
本实验旨在理解串口通信物理层原理,掌握USB转串口接口卡的连接方法,并实现基础的串口打开、关闭及数据的自收发功能。预期能够通过串口调试助手或自编程序,验证单机状态下的串口回路通信正常。

\subsection{实现思路}
% 硬件拓扑、功能流程、协议定义
\subsubsection{硬件拓扑}
% 绘制清晰的设备连接图(标注设备ID、COM口编号、链路类型)

\subsubsection{功能流程}
% 使用流程图描述核心逻辑

\subsubsection{协议定义}
% 明确数据帧格式、交互规则(单机实验主要涉及物理层配置)

\subsection{模块设计}
% 模块划分、调用关系、接口定义
\subsubsection{模块划分}

\subsubsection{接口定义}

\subsection{功能实现}
% 核心代码、实现效果、性能分析
\subsubsection{核心代码}

\subsubsection{实现效果}
% 附实验测试截图

\subsubsection{性能分析}

\section{实验二:双机通信实验(C/S模式)}

\subsection{功能要求}
% 简述对应实验的核心目标、需实现的功能点、预期效果
本实验要求掌握双机串口通信的交叉连接方法,理解并实现C/S(客户端/服务器)模式的通信逻辑。预期实现客户端发起请求,服务器端接收并处理请求后返回响应,且双方均能正确显示交互数据。

\subsection{实现思路}
% 硬件拓扑、功能流程、协议定义
\subsubsection{硬件拓扑}

\subsubsection{功能流程}

\subsubsection{协议定义}

\subsection{模块设计}
% 模块划分、调用关系、接口定义
\subsubsection{模块划分}

\subsubsection{接口定义}

\subsection{功能实现}
% 核心代码、实现效果、性能分析
\subsubsection{核心代码}

\subsubsection{实现效果}

\subsubsection{性能分析}

\section{实验三:简单拓扑的多机通信实验(链路层)}

\subsection{功能要求}
% 简述对应实验的核心目标、需实现的功能点、预期效果
本实验旨在理解链路层帧传输原理与树形拓扑连接。需实现4台计算机(1台根节点,3台叶子节点)组成的网络中,任意两台设备间的直接数据交付,包括中转转发逻辑。

\subsection{实现思路}
% 硬件拓扑、功能流程、协议定义
\subsubsection{硬件拓扑}

\subsubsection{功能流程}

\subsubsection{协议定义}

\subsection{模块设计}
% 模块划分、调用关系、接口定义
\subsubsection{模块划分}

\subsubsection{接口定义}

\subsection{功能实现}
% 核心代码、实现效果、性能分析
\subsubsection{核心代码}

\subsubsection{实现效果}

\subsubsection{性能分析}

\section{实验四:跨链路的多机通信实验(网络层)}

\subsection{功能要求}
% 简述对应实验的核心目标、需实现的功能点、预期效果
本实验聚焦网络层路由选择原理。需构建包含2跳以上路径的多链路拓扑(6台设备),实现静态或动态路由算法,支持设备的动态接入与退出,确保任意两台设备间可通过最优路径通信。

\subsection{实现思路}
% 硬件拓扑、功能流程、协议定义
\subsubsection{硬件拓扑}

\subsubsection{功能流程}

\subsubsection{协议定义}

\subsection{模块设计}
% 模块划分、调用关系、接口定义
\subsubsection{模块划分}

\subsubsection{接口定义}

\subsection{功能实现}
% 核心代码、实现效果、性能分析
\subsubsection{核心代码}

\subsubsection{实现效果}

\subsubsection{性能分析}

\section{实验五:多机可靠传输实验(运输层)}

\subsection{功能要求}
在实验四动态路由的基础上实现端到端可靠传输:
\begin{itemize}
  \item 实现停等协议(Stop-and-Wait),提供 SYN/SYN-ACK 建链、数据包确认与超时重传。
  \item 引入 CRC32 校验检测比特错误,可模拟校验码篡改与丢包,验证鲁棒性。
  \item 发送端可多次重传直到收到 ACK 或超过最大重试次数,接收端按序交付并处理重复帧。
\end{itemize}
预期效果:在故意注入校验错误或丢包场景下,仍能依靠超时重传与 ACK 机制保证消息最终可靠送达或给出失败提示。

\subsection{实现思路}
\subsubsection{硬件拓扑}
沿用实验四的多链路网状拓扑,所有节点运行 \texttt{ReliableRouterNode},重用底层串口与路由设施。拓扑结构与实验四完全相同,仍为 wolf、win、mac、A、B 等节点的网状连接,每个节点配备多个串口与邻居节点建立点对点链路。拓扑示意中可标注一条测试路径用于可靠传输演示,例如 wolf→win→mac,验证跨多跳的端到端可靠传输能力。在这种网状拓扑中,底层的距离向量路由负责找到最优路径并自动适应链路变化,上层的可靠传输协议则负责在端到端之间提供确认、重传与校验机制,两者分工明确、协同工作,共同构建了完整的运输层服务。

\begin{figure}[htbp]
\centering
\includegraphics[width=0.85\textwidth]{pic/exp6_topology.drawio.png}
\caption{实验五硬件拓扑示意图}
\label{fig:exp5_topology}
\end{figure}

\subsubsection{功能流程}
可靠传输系统的工作流程在距离向量路由的基础上增加了运输层的会话建立、数据确认、超时重传与校验机制,形成了端到端的可靠通信闭环。

\textbf{路由初始化} → 节点启动时,首先选择多个串口并设置本机 ID,完成路由层的初始化,包括邻居发现、路由表初始化等,同时启动 HELLO 和 DV 的周期性广播任务,确保底层路由持续工作;同时初始化可靠传输层的序号生成器、ACK 事件表与校验错误/丢包注入标志。

\textbf{会话建立}(SYN/SYN-ACK)→ 用户输入 \texttt{send <ID> <Msg>} 发送消息时,发送端进入运输层的可靠发送流程,首先为本次会话随机生成起始序号,构造运输层 SYN 帧(含源端口、目的端口、序号、CRC32 校验、类型标识与载荷),通过网络层查表获取下一跳端口,随后发送 SYN 到下一跳,同时启动超时定时器等待 SYN-ACK 响应。

\textbf{接收方会话确认} → 接收端收到 SYN 帧后,首先校验 CRC32 码,若校验通过则记录会话信息(源、目的、序号),回复 SYN-ACK 确认建链,否则静默丢弃该帧,等待发送端重传。

\textbf{数据帧发送与确认} → 发送端收到 SYN-ACK 后确认会话建立成功,随后封装数据帧(DAT 类型)包含序号递增值与消息负荷,按路由表发送到下一跳,同时启动等待定时器期望接收 ACK。接收端收到数据帧后,再次校验 CRC32,若校验通过且序号正确(无重复)则交付应用层供用户查看,并立即回复 ACK 确认;若校验失败或序号重复则静默丢弃该帧,依靠发送端的超时重传机制。

\textbf{超时重传机制} → 若发送端在预定超时时间内(如 2 秒)未收到期望的 ACK,则重传当前帧,重传次数达到最大上限(如 5 次)后放弃并返回失败,向用户提示发送失败,避免永久阻塞。

\textbf{错误与丢包模拟} → 为验证可靠传输的鲁棒性,程序提供 \texttt{corrupt on/off} 和 \texttt{loss on/off} 命令,分别模拟校验码篡改与丢包场景,\texttt{corrupt on} 时下一次发送的帧的校验码会被故意修改,接收端校验失败后丢弃该帧,发送端超时后自动重传正确的帧;\texttt{loss on} 时下一次发送的帧被直接丢弃而不实际发送,同样触发发送端的超时重传,终端输出明确的重传次数与成功或失败的最终结果。

\subsubsection{协议定义}
网络层保持 \texttt{DATA|Src|Dst|Payload},运输层内部帧格式:\texttt{SrcPort|DstPort|Seq|Checksum|Type|Body},其中 \texttt{Type} $\in$ \{\texttt{SYN}, \texttt{SAK}, \texttt{DAT}, \texttt{ACK}\}。校验使用 CRC32 计算 \texttt{Src|Dst|Seq|Type|Body} 的 32bit 值。

\subsection{模块设计}
\subsubsection{模块划分}
\begin{itemize}
  \item \textbf{路由层}:沿用实验四的邻居管理、距离向量、转发与超时检测。
  \item \textbf{可靠传输层}:在 \texttt{Code\_Refactored/Experiment5/reliable\_router.py} 中实现序号管理、ACK 事件同步、CRC 计算、超时重传与错误/丢包模拟。
  \item \textbf{用户接口}:\texttt{send}、\texttt{table}、corrupt on/off、loss on/off、\texttt{exit}。
\end{itemize}

\subsubsection{接口定义}
\begin{itemize}
  \item \texttt{\_initiate\_reliable\_send(target, msg)}:执行停等发送,含 SYN 建链、ACK 等待、超时重传与最大重试控制。
  \item \texttt{\_transport\_send\_ack(target, seq, is\_syn\_ack)}:发送 ACK/SYN-ACK 响应。
  \item \texttt{\_on\_recv\_data(src, dst, payload)}:校验、判重、按序交付并回复 ACK;若非目的节点则按路由转发。
\end{itemize}

\subsection{功能实现}
\subsubsection{核心代码}
发送方为每次会话随机生成起始序号,构造 \texttt{SYN} 帧并等待 \texttt{SYN-ACK};收到正确 ACK 后进入数据帧发送,未在超时内收到期望 ACK 则重传,直至成功或超过最大重试次数。接收方校验失败直接丢弃,校验通过后根据序号与类型决定回 ACK、丢弃重复帧或等待缺失帧。支持一次性模拟校验码错误或丢包以观测重传过程。

\subsubsection{实现效果}
在网状拓扑环境下,可靠传输协议的各项功能均达到预期效果。

\textbf{正常链路传输} → wolf 节点向 mac 节点发送消息,在无错误无丢包的情况下,单条消息一次发送即成功,SYN 和 SYN-ACK 快速完成会话建立,数据帧发送后立即收到 ACK,终端输出显示重传次数为 0,消息成功到达 mac 节点并被应用层接收,证明了正常情况下高效的端到端传输。

\textbf{CRC32 校验错误恢复} → 启用 \texttt{corrupt on} 后发送消息,首次发送的数据帧的 CRC32 校验码被故意篡改,接收端校验失败后静默丢弃该帧,发送端等待超时(2 秒)后自动重传,第二次发送的帧校验码正确,接收端成功接收并回复 ACK,发送端输出日志显示"重传 1 次成功",消息最终成功送达,验证了 CRC32 校验与超时重传机制的有效性。

\textbf{丢包检测与恢复} → 启用 \texttt{loss on} 后发送消息,首次发送的数据帧被直接丢弃而不实际发送,发送端等待超时后自动重传,第二次发送的帧正常到达,接收端成功接收并回复 ACK,发送端输出日志显示"重传 1 次成功",消息最终送达,证明了丢包场景下的恢复能力。

\textbf{多次重传与最大重试限制} → 连续启用多次 \texttt{loss on},发送端会多次重传直到成功或达到最大重试次数(5 次),若达到最大重试次数则返回失败并提示用户"发送失败:达到最大重试次数",避免永久阻塞,保证了系统的有界性与稳定性。

\begin{figure}[htbp]
\centering
\includegraphics[width=0.9\textwidth]{pic/exp5_results.png}
\caption{实验五实现效果截图}
\label{fig:exp5_results}
\end{figure}

\subsubsection{性能分析}
停等协议吞吐受 RTT 与重传次数影响,适合小规模控制报文。CRC32 计算与事件等待开销极低,主要开销来自可能的多次重传与串口速率。重传上限防止永远阻塞。

\paragraph{AI 提示词} 可靠传输实现时的提示词示例:\emph{“在现有串口 DV 路由器上增加停等可靠传输,设计包含 SYN/SYN-ACK、数据帧、ACK、CRC32 校验、超时重传与错误/丢包模拟的 Python 代码。”}

\section{实验六:简单网络管理实验(应用层)}

\subsection{功能要求}
在前述可靠多机通信的基础上实现应用层网络管理:
\begin{itemize}
  \item 实现 \texttt{ping}:检测可达性并测量 RTT。
  \item 实现 \texttt{traceroute}:通过逐步增加 TTL 获取路径中间节点。
  \item 增强网络层:引入 TTL 处理与 ICMP 报文,支持 Time Exceeded 反馈;支持可选可视化日志上报与远程命令下发。
\end{itemize}

\subsection{实现思路}
\subsubsection{硬件拓扑}
沿用实验四/五的多串口网状拓扑,所有节点运行 \texttt{NetworkNode},可任选一台作为前端可视化上报节点。拓扑结构仍为 wolf、win、mac、A、B 等节点的网状连接,每个节点配备多个串口与邻居节点建立点对点链路,形成多跳转发路径。拓扑示意中可标出负责可视化上报的节点(例如 wolf 节点配置了可视化后端地址),该节点周期性地将本地的邻居表、路由表与日志信息上报到后端服务器,同时接收后端下发的远程命令并执行,实现了网络管理的可视化与远程控制能力。其他节点则专注于网络层的 TTL 处理、ICMP 报文处理与路由转发,共同构建了支持应用层管理工具的完整网络环境。

\begin{figure}[htbp]
\centering
\includegraphics[width=0.85\textwidth]{pic/exp6_topology.drawio.png}
\caption{实验六硬件拓扑示意图}
\label{fig:exp6_topology}
\end{figure}

\subsubsection{功能流程}
应用层网络管理系统的工作流程在前述路由与可靠传输的基础上增加了 TTL 处理、ICMP 报文生成与解析、可视化上报等机制,形成了完整的网络管理闭环。

\textbf{节点初始化与路由维护} → 节点启动时,首先选择多个串口、设置本机 ID,并可选配置可视化后端地址(如 \url{http://localhost:8080}),完成路由层的初始化并启动 HELLO 和 DV 的周期性广播任务;后台任务持续周期发送 HELLO/DV 报文以维护路由表,同时检测邻居超时。

\textbf{可视化上报与远程命令接收} → 配置了可视化后端的节点,周期性(如每 5 秒)调用 REST 接口上报本地邻居表、路由表与日志信息到后端服务器,同时向后端查询是否有待执行的远程命令,接收到命令后(如 \texttt{ping mac}、\texttt{tracert A}、\texttt{send B Hello} 等)使用本地命令执行器处理,将执行结果返回后端,实现了远程控制与可视化管理的闭环。

\textbf{ICMP Ping 探测}(\texttt{ping <ID>} 命令)→ 用户执行 \texttt{ping mac 4} 发送 4 次 Echo Request,每次节点封装 ICMP Echo Request 报文(含序号、时间戳),初始 TTL 设为较大值(如 16),通过网络层路由发送到目标节点,同时记录发送时间戳并等待 Echo Reply 事件;目标节点收到 Echo Request 后,立即封装 Echo Reply 报文(携带原始时间戳和接收时间戳)返回源节点,源节点收到 Echo Reply 后计算 RTT(往返时延 = 当前时间 - 原始时间戳)并输出;多次 ping 完成后统计平均 RTT 与丢包率。

\textbf{ICMP Traceroute 探测}(\texttt{tracert <ID>} 命令)→ 用户执行 \texttt{tracert mac},节点从 TTL=1 开始逐步递增发送 ICMP Echo Request 探测包,每个中间路由节点在转发时发现 TTL 递减至 0 后,立即调用 \texttt{\_send\_icmp\_time\_exceeded} 返回 Time Exceeded 报文给源节点,源节点收到 Time Exceeded 时记录该跳路由的 ID 与时延,继续递增 TTL 发送下一跳探测(TTL=2、3、...),直至收到目标节点的 Echo Reply 或达到最大跳数(如 16 跳)为止,最终输出完整的路由路径列表(如 \texttt{1: win (RTT 15ms), 2: mac (RTT 30ms)})。

\textbf{常规消息与 ICMP 并行工作} → 常规消息仍可通过 \texttt{send <ID> <msg>} 发送,复用运输层的可靠传输通道(\texttt{PROTO\_TRANSPORT}),与 ICMP 报文在网络层共存互不干扰;用户可通过 \texttt{table} 命令查看当前路由表状态。

\textbf{网络异常处理} → TTL 每转发一跳递减 1,降至 0 时触发 Time Exceeded 返还源端;链路断开时,目标不可达则应用层工具显示超时或不可达提示;链路恢复后重新执行 ping/tracert 命令则能够正常输出结果。

\textbf{优雅退出} → 用户输入 \texttt{exit} 命令后,程序停止所有后台任务(包括可视化上报、HELLO/DV 广播、邻居超时检测等),逐个关闭所有串口资源,最终优雅终止。

\subsubsection{协议定义}
网络层数据包:\texttt{DATA|Src|Dst|TTL|Payload}。\texttt{Payload} 按子协议分发:
\begin{itemize}
  \item \texttt{ICMP}:\texttt{ICMP|ECHO\_REQ|Seq|Ts}、\texttt{ICMP|ECHO\_REP|Seq|OrigTs|RecvTs}、\texttt{ICMP|TIME\_EXC|Seq|RouterID}。
  \item \texttt{PROTO\_TRANSPORT}:向下兼容实验五的可靠消息载荷。
\end{itemize}
TTL 每转发一跳减 $1$,降至 $0$ 时触发 Time Exceeded 返还源端。

\subsection{模块设计}
\subsubsection{模块划分}
\begin{itemize}
  \item \textbf{网络与 ICMP 核心}(\texttt{Code\_Refactored/Experiment6/network\_app.py}):\texttt{NetworkNode} 处理 TTL、ICMP 逻辑、路由表维护及串口监听。
  \item \textbf{可视化上报}:周期调用 REST 接口上报邻居、路由与日志,可接受后端下发的远程命令并重用本地命令执行器。
  \item \textbf{事件管理}:\texttt{icmp\_events/icmp\_results} 记录序列号对应的等待与结果,驱动 ping/traceroute 阻塞等待。
\end{itemize}

\subsubsection{接口定义}
\begin{itemize}
  \item \texttt{do\_ping(target, count)}:发送多次 Echo Request,等待事件,统计 RTT 与丢包率。
  \item \texttt{do\_traceroute(target)}:递增 TTL 发送探测包,收集每跳的 Time Exceeded 或最终 Echo Reply。
  \item \texttt{\_process\_network\_packet(src, dst, ttl, payload)}:处理 TTL、转发或交付到 ICMP/可靠消息分发。
\end{itemize}

\subsection{功能实现}
\subsubsection{核心代码}
TTL 在进入 \texttt{\_process\_network\_packet} 时递减,耗尽即调用 \texttt{\_send\_icmp\_time\_exceeded} 返回源端。ICMP Echo Reply 中携带原始时间戳以计算 RTT;Time Exceeded 返回触发 traceroute 的每跳记录。串口发送均加锁避免并发写冲突,端口异常会被关闭并从邻居表移除。

\subsubsection{实现效果}
在 6 节点网络中,应用层网络管理工具的各项功能均运行稳定且效果显著。

\textbf{ICMP Ping 测试} → wolf 节点执行 \texttt{ping mac 4} 发送 4 次 Echo Request,每次往返的 RTT 被准确计算并输出(如 RTT: 150ms、145ms、152ms、148ms),最终统计平均 RTT 为 148.75ms,丢包率为 0\%,验证了 ICMP Echo 机制的正确性与时间戳计算的准确性;若目标节点不可达,则显示超时或不可达提示,用户清晰了解网络连接状态。

\textbf{Traceroute 路由发现} → wolf 节点执行 \texttt{tracert mac},程序从 TTL=1 开始逐跳探测,第 1 跳收到 win 节点返回的 Time Exceeded(延迟 15ms),第 2 跳收到 mac 节点返回的 Echo Reply(延迟 30ms),最终输出完整路由路径:\texttt{1: win (15ms), 2: mac (30ms)},验证了 TTL 递减与 Time Exceeded 机制的有效性,用户可清晰了解从源到目标的完整转发路径。

\textbf{链路故障与恢复} → 拔掉某条链路后执行 \texttt{ping} 或 \texttt{tracert},若目标不可达则显示超时或不可达提示;若链路恢复后重新执行 ping 或 tracert,则能够重新发现新路径或恢复到原有路径,证明了应用层工具与底层动态路由的紧密配合与自适应能力。

\textbf{可视化管理与远程控制} → 若配置了后端地址,wolf 节点周期性上报邻居表、路由表与日志信息到后端服务器,管理员通过 Web 界面可实时查看网络拓扑、路由表变化与日志流;同时可远程触发 \texttt{ping}、\texttt{tracert}、\texttt{send}、\texttt{table} 等命令并查看执行结果,实现了网络管理的可视化与远程控制。

\begin{figure}[htbp]
\centering
\includegraphics[width=0.9\textwidth]{pic/exp6_results.png}
\caption{实验六实现效果截图}
\label{fig:exp6_results}
\end{figure}

\subsubsection{性能分析}
TTL 与 ICMP 处理逻辑轻量,瓶颈仍在串口速率与路由收敛时间。事件等待以秒级超时控制,避免阻塞。可视化上报设置 0.5s 超时,不阻塞主流程。

\paragraph{AI 提示词} 应用层工具设计时的提示词示例:\emph{“在已有串口 DV 路由器上实现 ICMP 支持,包括 TTL 递减、Time Exceeded、Echo Request/Reply,并提供 ping 和 traceroute 命令,可选将路由与日志上报到 HTTP 接口。”}

\section{总结与感想}

\subsection{核心收获}
通过从单机回环到应用层管理工具的六个递进实验,本课程设计完整走了一遍从物理层到应用层的实现路径,直观体会了分层解耦、协议设计和状态维护在网络通信中的重要作用。实验一的单机回环通信使我们掌握了串口配置、数据收发与缓冲区管理的基本技能,建立了对物理层与数据链路层的直观认识。实验二的双机 C/S 通信引入了应用层协议的概念,通过命令解析与响应机制体会了协议设计的灵活性与可扩展性。实验三的树形拓扑多机通信实现了基于 ID 的帧转发,初步建立了网络层寻址与转发的概念。实验四的距离向量路由算法是整个课程设计的核心,通过 Bellman-Ford 方程、毒性逆转与邻居超时检测等机制,深刻理解了分布式路由的工作原理与自适应能力,亲手实现了路由表的动态收敛、链路故障检测与路由恢复。实验五的停等可靠传输协议引入了 CRC32 校验、超时重传与会话建立机制,体会了运输层在端到端可靠通信中的关键作用,同时通过故意注入错误与丢包验证了协议的鲁棒性。实验六的 ICMP 与网络管理工具将理论知识转化为实用工具,ping 和 traceroute 的实现加深了对 TTL、ICMP 报文与多跳转发的理解,可视化上报与远程控制则通过 FastAPI 后端与 React 前端形成闭环,展示了应用层在网络管理中的可视化与远程协同价值。整个实验过程中,串口通信的粘包/分包处理、距离向量的收敛过程、停等协议的重传机制、TTL 递减与 ICMP 报文的生成解析等知识在实践中得到巩固,Python 多线程编程、事件同步与并发控制的工程能力也得到了显著提升。

\subsection{问题与解决方案}
实验过程中遇到了若干典型问题,通过分析与调试逐一解决,积累了宝贵的工程经验。串口粘包与半包问题在实验初期频繁出现,尤其是在高速率或长消息场景下,接收端可能一次读取多个完整帧或半个帧,导致解析错误。解决方案是统一采用换行符作为分帧边界,接收线程内部使用 \texttt{readline()} 或累积读取直到遇到完整的换行符,确保每次解析的都是完整的帧,必要时在接收端维护缓冲区以处理半包情况。路由环路与路由失效问题在实验四初期较为突出,未实现毒性逆转时,某些拓扑下会出现路由环路导致数据包无限转发,未实现邻居超时检测时,链路断开后路由表仍保留过时路由导致黑洞。解决方案是在距离向量通告中加入毒性逆转逻辑,向邻居通告时若某目的的下一跳是该邻居则谎报开销为 999,从根本上避免环路;同时实现邻居超时检测机制,10 秒内未收到 HELLO 则移除邻居并将相关路由开销置为 999,触发 DV 更新以通知全网。可靠传输的丢包与误码恢复在实验五中通过停等协议配合 CRC32 校验与超时重传得到验证,即使模拟篡改或丢包也能最终恢复或在达到最大重试次数后超时放弃,避免永久阻塞。串口拔插异常在多次实验中导致程序崩溃或残留后台线程,解决方案是在所有串口写入操作前检查串口状态,写入失败时捕获异常并关闭该端口,同时从邻居表与转发表中移除对应条目,确保程序鲁棒性。

\subsection{改进建议与展望}
基于当前实验的实现与测试,提出若干改进建议与未来展望,以进一步提升系统的性能与实用性。路由算法方面,当前的距离向量算法在小规模拓扑下表现良好,但在大规模拓扑下收敛时间较长且开销较大,可以扩展为链路状态算法(如 OSPF),每个节点维护全网拓扑并独立计算最短路径,收敛速度更快且支持更复杂的拓扑结构,同时可以引入度量加权(如带宽、延迟)以支持更灵活的路由策略。可靠传输方面,当前的停等协议吞吐受 RTT 限制,每次发送后必须等待 ACK 才能发送下一帧,效率较低,可以升级为滑动窗口协议(Go-Back-N 或 Selective Repeat),发送端可以连续发送多个帧而无需逐个等待 ACK,显著提高吞吐量并减少等待时间,同时支持流量控制与拥塞控制机制。应用层工具方面,当前的 ping 和 traceroute 已经实现了基本功能,可以进一步增加文件传输工具(如基于可靠传输的 FTP)、节点监控面板(实时显示 CPU、内存、端口状态)等实用功能,同时完善 Web 端可视化的安全认证机制(如 JWT)与数据持久化(如数据库存储历史路由表与日志),提升系统的实用性与安全性。所有实验源代码与报告已同步至 GitHub 仓库(\url{https://github.com/Spercent521/network-curriculum_design-CUG}),便于复现与进一步改进,欢迎其他同学与开发者参考与贡献代码。



% 参考文献部分
\printbibliography[title={参考文献}]

\end{document}
